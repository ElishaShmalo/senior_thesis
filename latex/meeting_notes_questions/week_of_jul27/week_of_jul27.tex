\documentclass{article}
\usepackage{amsmath}
\usepackage{amssymb}


\author{Elisha Shmalo}
\title{Notes: Week of Jul 21}

\begin{document}

\maketitle

\section{Derivation of Equations of Motion}

\subsection{Hamiltonian Dynamics}

Our basis:

\[
S_j^x = s_j, \space S_j^y = f(s_j)\cos(\phi_j), \space S_j^z = f(s_j)\sin(\phi_j)
\]

With $f(x) := \sqrt{1 - x^2}$.

The hamiltonian is

\[
H = \sum_j (-\vec{J} \circ \vec{S_{j+1}}) \cdot \vec{S_j}
\]

The only terms in $H$ involving $j$ values are

\begin{align*}
    (H)_j = &-J_x(s_js_{j-1} + s_js_{j+1}) \\
            &-J_yf(s_j)\cos\phi_j[f(s_{j-1})\cos\phi_{j-1} + f(s_{j+1})\cos\phi_{j+1}] \\
            &-J_zf(s_j)\sin\phi_j[f(s_{j-1})\sin\phi_{j-1} + f(s_{j+1})\sin\phi_{j+1}]
\end{align*}

The cononical position is

\[s_j\]

The cononical momentum is

\[\phi_j\]

Thus

\begin{align*}
    \dot{s_j} &= \frac{\partial H}{\partial \phi_j}\\
    \dot{\phi_j} &= -\frac{\partial H}{\partial s_j}
\end{align*}

and so we have

\begin{align*}
    \frac{ds_j}{dt} = &+J_yf(s_j)\sin(\phi_j)[f(s_{j+1})\cos(\phi_{j+1}) + f(s_{j-1})\cos(\phi_{j-1})] \\
                      &-J_zf(s_j)\cos(\phi_j)[f(s_{j+1})\sin(\phi_{j+1}) + f(s_{j-1})\sin(\phi_{j-1})]
\end{align*}

\begin{align*}
    \frac{d\phi_j}{dt} = -f'(s_j)[&-J_y\cos\phi_j(f(s_{j+1})\cos\phi_{j+1} + f(s_{j-1})\cos\phi_{j-1}) \\
                                  &-J_z\sin\phi_j(f(s_{j+1})\sin\phi_{j+1} + f(s_{j-1})\sin\phi_{j-1})] \\
                                  &+J_x(s_{j+1} + s_{j-1})
\end{align*}


\subsection{If $J_y = J_z = J$}

\begin{align*}
    \frac{ds_j}{dt} = &Jf(s_j)[f(s_{j+1}) (\sin\phi_j\cos\phi_{j+1} - \cos\phi_j\sin\phi_{j+1})\\
                      &+f(s_{j-1})(\sin\phi_j\cos\phi_{j-1} - \cos\phi_j\sin\phi_{j-1})]
\end{align*}

\begin{align*}
    \frac{d\phi_j}{dt} = Jf'(s_j)[&f(s_{j+1})(\cos\phi_j\cos\phi_{j+1} + \sin\phi_j\sin\phi_{j+1}) \\
                                  &+f(s_{j-1})(\cos\phi_j\cos\phi_{j-1} + \sin\phi_j\sin\phi_{j-1})] \\
                                  &+J_x(s_{j+1} + s_{j-1})
\end{align*}


which become

\begin{align*}
    \frac{ds_j}{dt} = &Jf(s_j)[f(s_{j+1})\sin(\phi_j - \phi_{j+1})\\
                      &+f(s_{j-1})\sin(\phi_j - \phi_{j-1})]
\end{align*}

\begin{align*}
    \frac{d\phi_j}{dt} = Jf'(s_j)[&f(s_{j+1})\cos(\phi_j - \phi_{j+1}) \\
                                  &+f(s_{j-1})\cos(\phi_j - \phi_{j-1})] \\
                                  &+J_x(s_{j+1} + s_{j-1})
\end{align*}

\subsection{Second order Expansion of functions}
\subsubsection{$f(x)$ and $f'(x)$}
Let 
\[
f(x) = \sqrt{1 - x^2}
\]
We expand \( f(\delta x) \) using a Taylor expansion:

\[
\boxed{f(\delta x) \approx 1 + 0 - 1/2 \delta x^2}
\]

The derivative:

\[
f'(x) = \frac{d}{dx} \left( \sqrt{1 - x^2} \right) = \frac{-x}{\sqrt{1 - x^2}}
\]

Then:

\[
f'(0) = 0
\]

\[
f''(0) = -1
\]

\[
f'''(0) = 0
\]

Thus:

\[
\boxed{f'(\delta x) \approx -\delta x}
\]

even up to second order.

\subsubsection{Now $\cos(\phi_j - \phi_{j\pm1})$ and $\sin{\phi_j - \phi_{j\pm1}}$}

Let
\[
\phi_j = \phi_j^0 + \delta \phi_j, \quad \text{where} \quad \phi_j^0 = \frac{2\pi}{N} j
\]

Let 

\[
x = \phi_j^0 - \phi_{j+1}^0, y = \phi_j^0 - \phi_{j-1}^0
\]

Then 
\[
    \cos(\phi_j - \phi_{j+1}) = \cos(x + \delta x)
\]

and similar for the others.

Note

\[
\cos(x) = \cos(y), \sin(x) = -\sin(y)
\]

Now for the exapnsions:

\[
    \boxed{\cos(x + \delta x) \approx \cos(x) - \sin(x)\delta x - 1/2\cos(x)\delta x^2}
\]

and 

\[
    \boxed{\sin(x + \delta x) \approx \sin(x) + \cos(x)\delta x - 1/2\sin(x)\delta x^2}
\]

\subsection{Second order approximation of diffrential equations}

\subsubsection{First $ds_j/dt$}

Recall

\begin{align*}
    \frac{ds_j}{dt} = &Jf(s_j)[f(s_{j+1})\sin(\phi_j - \phi_{j+1})\\
                      &+f(s_{j-1})\sin(\phi_j - \phi_{j-1})]
\end{align*}

Using the exapnsions

\begin{align*}
    \frac{ds_j}{dt} \approx &J(1-1/2\delta s_j^2)[(1-1/2\delta s_{j+1}^2)(\sin(x) + \cos(x)\delta x - 1/2\sin(x)\delta x^2)\\
                      &+(1-1/2\delta s_{j-1}^2)(\sin(y) + \cos(y)\delta y - 1/2\sin(y)\delta y^2)]
\end{align*}

Only keeping up to order two inside the brackets and grouping like terms gives us

\begin{align*}
    \frac{ds_j}{dt} \approx &J(1-\delta s_j^2)[\cos(x)(\delta x + \delta y) - 1/2\sin(x)(\delta x^2 - \delta y^2 + \delta s_{j+1}^2 - \delta s_{j-1}^2)]\\
\end{align*}

Again we are only keeping up to second order in $\delta ...$. So the terms with $-\delta s_j^2$ don't survive and we are left with 

\begin{align*}
    \frac{ds_j}{dt} \approx &J[\cos(x)(\delta x + \delta y) - 1/2\sin(x)(\delta x^2 - \delta y^2 + \delta s_{j+1}^2 - \delta s_{j-1}^2)]\\
\end{align*}

Which is

\begin{align*}
    \frac{ds_j}{dt} \approx J[&\cos(\frac{2\pi}{N})(2\delta\phi_{j} - \delta\phi_{j+1} - \delta\phi_{j-1}) \\
                                +&1/2\sin(\frac{2\pi}{N})(\delta\phi_{j+1}^2 - \delta\phi_{j-1}^2 + 2\delta\phi_{j}(\delta\phi_{j+1} - \delta\phi_{j-1}) + \delta s_{j+1}^2 - \delta s_{j-1}^2)]
\end{align*}

\subsubsection{Next $d\phi_j/dt$}

Recall

\begin{align*}
    \frac{d\phi_j}{dt} = Jf'(s_j)[&f(s_{j+1})\cos(\phi_j - \phi_{j+1}) \\
                                  &+f(s_{j-1})\cos(\phi_j - \phi_{j-1})] \\
                                  &+J_x(s_{j+1} + s_{j-1})
\end{align*}

using the expansions to second order

\begin{align*}
    \frac{d\phi_j}{dt} \approx -J\delta s_j[&(1-1/2\delta s_{j+1}^2)(\cos(x) - \sin(x)\delta x - 1/2 \cos(x)\delta x^2) \\
                                  &+(1-1/2\delta s_{j+1}^2)(\cos(y) - \sin(y)\delta y - 1/2 \cos(y)\delta y^2)] \\
                                  &+J_x(s_{j+1} + s_{j-1})
\end{align*}


When we consider that we are only keeping things up to second order a lot of terms don't survive

\begin{align*}
    \frac{d\phi_j}{dt} \approx -J\delta s_j[&(\cos(x) - \sin(x)\delta x) \\
                                  &+(\cos(y) - \sin(y)\delta y)] \\
                                  &+J_x(s_{j+1} + s_{j-1})
\end{align*}

Which simplifies to 

\begin{align*}
    \frac{d\phi_j}{dt} \approx -J\delta s_j[&2\cos(x) - \sin(x)(\delta x - \delta y)] \\
                                  &+J_x(s_{j+1} + s_{j-1})
\end{align*}

which is

\begin{align*}
    \frac{d\phi_j}{dt} \approx -J\delta s_j[&2\cos(\frac{2\pi}{N}) + \sin(\frac{2\pi}{N})(\delta\phi_{j-1} - \delta\phi_{j+1})] \\
                                  &+J_x(s_{j+1} + s_{j-1})
\end{align*}

\space

Thus we have that the equations of motion to second order are

\begin{align*}
    \frac{ds_j}{dt} \approx J[&\cos(\frac{2\pi}{N})(2\delta\phi_{j} - \delta\phi_{j+1} - \delta\phi_{j-1}) \\
                                +&1/2\sin(\frac{2\pi}{N})(\delta\phi_{j+1}^2 - \delta\phi_{j-1}^2 + 2\delta\phi_{j}(\delta\phi_{j+1} - \delta\phi_{j-1}) + \delta s_{j+1}^2 - \delta s_{j-1}^2)]
\end{align*}

\begin{align*}
    \frac{d\phi_j}{dt} \approx -J\delta s_j[&2\cos(\frac{2\pi}{N}) + \sin(\frac{2\pi}{N})(\delta\phi_{j-1} - \delta\phi_{j+1})] \\
                                  &+J_x(s_{j+1} + s_{j-1})
\end{align*}

\section{Seeking Solitons}

Try 

\begin{align*}
    s_j &= f(z), z = j - ut \\
    \phi_j &= \frac{2 \pi}{N} + g(z)
\end{align*}

Such that 

\begin{align*}
    \delta s_j &= f(z), z = j - ut \\
    \delta \phi_j &= g(z)
\end{align*}

Rewriting the second order approximation for the equations of motion:

\begin{align*}
    -uf'(z)\approx J[&\cos(\frac{2\pi}{N})(2g(z) - g(z+1) - g(z-1)) \\
                                +&1/2\sin(\frac{2\pi}{N})(g(z+1)^2 - g(z-1)^2 + 2g(z)(g(z+1) - g(z-1)) + f(z+1)^2 - f(z-1)^2)]
\end{align*}

\begin{align*}
    -ug'(z) \approx -Jf(z)[&2\cos(\frac{2\pi}{N}) + \sin(\frac{2\pi}{N})(g(z-1) - g(z+1))] \\
                                  &+J_x(f(z+1) + f(z-1))
\end{align*}

If we take the continuam limit and a linear exapnsions

\begin{align*}
    f(z \pm 1) &\approx f(z) \pm f'(z) \\
    g(z \pm 1) &\approx g(z) \pm g'(z)
\end{align*}

Then we get the follwing relations

\begin{align*}
    &2g(z) - g(z+1) - g(z-1) \approx 0 \\
    &g(z-1) - g(z+1) \approx -2g' \\
    &f(z - 1) + f(z + 1) \approx 2f \\
    &g(z)(g(z+1) - g(z-1)) \approx 2gg' \\
    &g(z+1)^2 - g(z-1)^2 = (g(z+1) + g(z - 1))(g(z+1) - g(z-1)) \approx 2g2g' = 4gg' \\
    &f(z+1)^2 - f(z-1)^2 \approx 4ff' \\
\end{align*}

(where I've written $g(z) = g$ and $f(z) = f$ for brevity)

Using these (and defining $\theta = \frac{2\pi}{N}$), our diffrential equations become

\begin{align*}
    -uf' &\approx J[0 +1/2\sin(\theta)(4gg' + 4gg' + 4ff')] \\
            &= 2J\sin(\theta)[2gg' + ff']
\end{align*}

\begin{align*}
    -ug' \approx -Jf[2\cos(\theta) - 2\sin(\theta)g'] + 2J_xf
\end{align*}

Rearanging these become

\begin{align*}
    f' &\approx -4J\sin(\theta)\frac{gg'}{u + 2J\sin(\theta)f} \\
    g' &\approx \frac{2(J\cos(\theta) - J_x)}{u + 2J\sin(\theta)f}f
\end{align*}

\section{Harmonic Oscillators to Model Spin}

Consider a series RLC circuit driven by an AC voltage source:
\[
V_{\text{drive}}(t) = V_0 \cos(\omega_d t)
\]
The circuit contains:
\begin{itemize}
  \item Inductor with inductance $L$
  \item Resistor with resistance $R$
  \item Capacitor with capacitance $C$
\end{itemize}

\subsection{Equation of Motion}
\[
V_L + V_R + V_C = V_{\text{drive}}(t)
\]
\[
\boxed{L \ddot{Q} + R \dot{Q} + \frac{1}{C} Q = V_0 \cos(\omega_d t)}
\]

\subsection{Solution Structure}

The general solution is:
\[
Q(t) = Q_{\text{hom}}(t) + Q_{\text{part}}(t)
\]

The homogeneous solution describes transient behavior and decays over time due to resistance with time scale.

\[\tau = -\frac{2L}{R}\].

As in 

\[Q_{\text{hom}}(t) \sim e^{-\frac{Rt}{2L}}\]

I seek a steady state solution, so I feel like we should try something like:
\[
Q_{\text{part}}(t) = A \cos(\omega_d t) + B \sin(\omega_d t)
\]

\subsection{Determaning Particular Solution Ansatz}
Compute the first and second derivatives:
\[
\dot{Q}_{\text{part}}(t) = -A \omega_d \sin(\omega_d t) + B \omega_d \cos(\omega_d t)
\]
\[
\ddot{Q}_{\text{part}}(t) = -A \omega_d^2 \cos(\omega_d t) - B \omega_d^2 \sin(\omega_d t)
\]

Substitute into the differential equation:
\begin{align*}
&L \ddot{Q}_{\text{part}} + R \dot{Q}_{\text{part}} + \frac{1}{C} Q_{\text{part}} = V_0 \cos(\omega_d t) \\
&= L(-A \omega_d^2 \cos(\omega_d t) - B \omega_d^2 \sin(\omega_d t)) \\
&\quad + R(-A \omega_d \sin(\omega_d t) + B \omega_d \cos(\omega_d t)) \\
&\quad + \frac{1}{C}(A \cos(\omega_d t) + B \sin(\omega_d t))
\end{align*}

Group terms:
\[
\text{Coefficient of } \cos(\omega_d t):\quad -L A \omega_d^2 + R B \omega_d + \frac{A}{C}
\]
\[
\text{Coefficient of } \sin(\omega_d t):\quad -L B \omega_d^2 - R A \omega_d + \frac{B}{C}
\]

Set the equation equal to the driving term \( V_0 \cos(\omega_d t) \), and match coefficients:
\[
\begin{cases}
- L A \omega_d^2 + R B \omega_d + \frac{A}{C} = V_0 \\
- L B \omega_d^2 - R A \omega_d + \frac{B}{C} = 0
\end{cases}
\]

\subsection*{Solve the System}

We now solve this linear system for \( A \) and \( B \).

Define:
\[
X := \left( \frac{1}{C} - L \omega_d^2 \right), \quad Y := R \omega_d
\]

Then the system becomes:
\[
\begin{cases}
A X + B Y = V_0 \\
B X - A Y = 0
\end{cases}
\]

Solve the second equation for \( B \):
\[
B X = A Y \Rightarrow B = \frac{A Y}{X}
\]

Substitute into the first equation:
\[
A X + \left( \frac{A Y}{X} \right) Y = V_0
\Rightarrow A \left( X + \frac{Y^2}{X} \right) = V_0
\Rightarrow A = \frac{V_0 X}{X^2 + Y^2}
\]

Then:
\[
B = \frac{A Y}{X} = \frac{V_0 Y}{X^2 + Y^2}
\]

\subsection*{Final Particular Solution}

Thus, the particular solution is:
\[
Q_{\text{part}}(t) = \frac{V_0 X}{X^2 + Y^2} \cos(\omega_d t) + \frac{V_0 Y}{X^2 + Y^2} \sin(\omega_d t)
\]
\[
\text{where } X = \left( \frac{1}{C} - L \omega_d^2 \right), \quad Y = R \omega_d
\]

This can also be written in amplitude-phase form:
\[
Q_{\text{part}}(t) = Q_p \cos(\omega_d t - \delta)
\]

with:
\[
Q_p = \frac{V_0}{\sqrt{X^2 + Y^2}} = \frac{V_0}{\sqrt{\left( \frac{1}{C} - L \omega_d^2 \right)^2 + (R \omega_d)^2}}
\]
\[
\tan \delta = \frac{Y}{X} = \frac{R \omega_d}{\frac{1}{C} - L \omega_d^2}
\]

\subsection*{Voltage Across the Capacitor}

The voltage across the capacitor is:
\[
V_C(t) = \frac{Q_{\text{part}}(t)}{C} = \frac{Q_p}{C} \cos(\omega_d t - \delta)
\]

Final expression:
\[
\boxed{
V_C(t) = \frac{V_0}{\sqrt{(1 - L C \omega_d^2)^2 + (R C \omega_d)^2}} \cos(\omega_d t - \delta)
}
\quad \text{with} \quad
\tan \delta = \frac{R \omega_d}{\frac{1}{C} - L \omega_d^2}
\]

\end{document}