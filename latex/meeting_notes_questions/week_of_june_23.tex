\documentclass{article}
\usepackage{amsmath}
\usepackage{amssymb}
\usepackage[margin=1in]{geometry} 

\author{Elisha Shmalo}
\title{Notes: Week of June 23}

\begin{document}

\maketitle

\section{Analytic derivatation of Lyop-Exp}

\subsection{Hamiltonian Dynamics}

Our basis:

\[
S_j^x = s_j, \space S_j^y = f(s_j)\cos(\phi_j), \space S_j^z = f(s_j)\sin(\phi_j)
\]

With $f(x) := \sqrt{1 - x^2}$.

The hamiltonian is

\[
H = \sum_j (-\vec{J} \circ \vec{S_{j+1}}) \cdot \vec{S_j}
\]

The only terms in $H$ involving $j$ values are

\begin{align*}
    (H)_j = &-J_x(s_js_{j-1} + s_js_{j+1}) \\
            &-J_yf(s_j)\cos\phi_j[f(s_{j-1})\cos\phi_{j-1} + f(s_{j+1})\cos\phi_{j+1}] \\
            &-J_zf(s_j)\sin\phi_j[f(s_{j-1})\sin\phi_{j-1} + f(s_{j+1})\sin\phi_{j+1}]
\end{align*}

The cononical position is

\[s_j\]

The cononical momentum is

\[\phi_j\]

Thus

\begin{align*}
    \dot{s_j} &= \frac{\partial H}{\partial \phi_j}\\
    \dot{\phi_j} &= -\frac{\partial H}{\partial s_j}
\end{align*}

and so we have

\begin{align*}
    \frac{ds_j}{dt} = &+J_yf(s_j)\sin(\phi_j)[f(s_{j+1})\cos(\phi_{j+1}) + f(s_{j-1})\cos(\phi_{j-1})] \\
                      &-J_zf(s_j)\cos(\phi_j)[f(s_{j+1})\sin(\phi_{j+1}) + f(s_{j-1})\sin(\phi_{j-1})]
\end{align*}

\begin{align*}
    \frac{d\phi_j}{dt} = -f'(s_j)[&-J_y\cos\phi_j(f(s_{j+1})\cos\phi_{j+1} + f(s_{j-1})\cos\phi_{j-1}) \\
                                  &-J_z\sin\phi_j(f(s_{j+1})\sin\phi_{j+1} + f(s_{j-1})\sin\phi_{j-1})] \\
                                  &-J_x(s_{j+1} + s_{j-1})
\end{align*}

\subsection{First order Expansion of functions}
\subsubsection{$f(x)$ and $f'(x)$}
Let 
\[
f(x) = \sqrt{1 - x^2}
\]
We expand \( f(x_0 + \delta x) \) to leading order in \( \delta x \) using a Taylor expansion:

\[
f(x_0 + \delta x) \approx f(x_0) + f'(x_0) \, \delta x
\]

The derivative:

\[
f'(x) = \frac{d}{dx} \left( \sqrt{1 - x^2} \right) = \frac{-x}{\sqrt{1 - x^2}}
\]

Then:

\[
f'(x_0) = \frac{-x_0}{\sqrt{1 - x_0^2}}
\]

Thus:

\[
f(x_0 + \delta x) \approx \sqrt{1 - x_0^2} - \frac{x_0}{\sqrt{1 - x_0^2}} \, \delta x
\]

For $x_0 = 0$, \[\boxed{f(x_0 + \delta x) \approx 1}\]

Simiarly

\[\boxed{f'(x_0 + \delta x) \approx -\delta x}\]

\subsubsection{$cos\phi_j$ and $sin\phi_j$}

Let
\[
\phi_j = \phi_j^0 + \delta \phi_j, \quad \text{where} \quad \phi_j^0 = \frac{2\pi}{N} j
\]

Using the expansion:
\[
\cos(a + \epsilon) \approx \cos a - \sin a \cdot \epsilon
\]

We have 

\[
\boxed{
\cos\phi_j \approx \cos( \frac{2\pi}{N}j) - \sin( \frac{2\pi}{N}j)\delta\phi_j
}
\]

For $N = 4$

\[\cos\phi_j \approx
\begin{cases}
    \mp \delta\phi_j, & j \text{odd} \\
    \pm 1, & j \text{even}
\end{cases}
\]

Simiarly using the expansion:
\[
\sin(a + \epsilon) \approx \sin a + \cos a \cdot \epsilon
\]

We have 

\[
\boxed{
\sin\phi_j \approx \sin( \frac{2\pi}{N}j) + \cos( \frac{2\pi}{N}j)\delta\phi_j
}
\]

For $N = 4$

\[\sin\phi_j 
\begin{cases}
    \pm 1, & j \text{odd} \\
    \pm \delta\phi_j, & j \text{even}
\end{cases}\]


\subsubsection{$cos(\phi_{j+1} - \phi_j)$}

In the notes I have:

\[``cos(\phi_{j+1} - \phi_j) \approx 1 + O(\delta \phi^2),'' N = 4\]

Let
\[
\phi_j = \phi_j^0 + \delta \phi_j, \quad \text{where} \quad \phi_j^0 = \frac{2\pi}{N} j
\]
with \( N \in \mathbb{Z} \). Then:
\[
\phi_{j+1} - \phi_j = \left( \phi_{j+1}^0 - \phi_j^0 \right) + \left( \delta \phi_{j+1} - \delta \phi_j \right) = \frac{2\pi}{N} + (\delta \phi_{j+1} - \delta \phi_j)
\]

We expand the cosine to first order in \( \delta \phi_j \) and \( \delta \phi_{j+1} \):

\[
\cos(\phi_{j+1} - \phi_j) = \cos\left( \frac{2\pi}{N} + \delta \phi_{j+1} - \delta \phi_j \right)
\]

Using the expansion:
\[
\cos(a + \epsilon) \approx \cos a - \sin a \cdot \epsilon
\]

with \( a = \frac{2\pi}{N} \) and \( \epsilon = \delta \phi_{j+1} - \delta \phi_j \), we obtain:

\[
\boxed{
\cos(\phi_{j+1} - \phi_j) \approx \cos\left( \frac{2\pi}{N} \right) - \sin\left( \frac{2\pi}{N} \right)(\delta \phi_{j+1} - \delta \phi_j) + O(\delta \phi ^2)
}
\]

If $N = 4$, 

\[
\cos(\phi_{j+1} - \phi_j) \approx  -(\delta \phi_{j+1} - \delta \phi_j)
\]

\subsubsection{$sin(\phi_{j+1} - \phi_j)$}

Let
\[
\phi_j = \phi_j^0 + \delta \phi_j, \quad \text{where} \quad \phi_j^0 = \frac{2\pi}{N} j
\]
Then:
\[
\phi_{j+1} - \phi_j = \frac{2\pi}{N} + (\delta \phi_{j+1} - \delta \phi_j)
\]

We expand the sine to first order in \( \delta \phi_j \) and \( \delta \phi_{j+1} \):

\[
\sin(\phi_{j+1} - \phi_j) = \sin\left( \frac{2\pi}{N} + \delta \phi_{j+1} - \delta \phi_j \right)
\]

Using the Taylor expansion:
\[
\sin(a + \epsilon) \approx \sin a + \cos a \cdot \epsilon
\]

with \( a = \frac{2\pi}{N} \) and \( \epsilon = \delta \phi_{j+1} - \delta \phi_j \), we obtain:

\[
\boxed{
\sin(\phi_{j+1} - \phi_j) \approx \sin\left( \frac{2\pi}{N} \right) + \cos\left( \frac{2\pi}{N} \right)(\delta \phi_{j+1} - \delta \phi_j) + O(\delta \phi ^2)
}
\]

If $N = 4$, Then

\[
\sin(\phi_{j+1} - \phi_j) \approx 1 + O(\delta \phi ^2)
\]

\subsection{First order approximation of diffrential equations}

Let 

\[
s_j = s_j^0 + \delta s_j = \delta s_j \quad \phi_j = \phi_j^0 + \delta\phi_j
\]

Recall the equations of motion in our chosen basis

\begin{align*}
    \frac{ds_j}{dt} = \frac{d\delta s_j}{dt} = &+J_yf(s_j)\sin(\phi_j)[f(s_{j+1})\cos(\phi_{j+1}) + f(s_{j-1})\cos(\phi_{j-1})] \\
                      &-J_zf(s_j)\cos(\phi_j)[f(s_{j+1})\sin(\phi_{j+1}) + f(s_{j-1})\sin(\phi_{j-1})]
\end{align*}

\begin{align*}
    \frac{d\phi_j}{dt} = \frac{d\delta\phi_j}{dt}  = -f'(s_j)[&-J_y\cos\phi_j(f(s_{j+1})\cos\phi_{j+1} + f(s_{j-1})\cos\phi_{j-1}) \\
                                  &-J_z\sin\phi_j(f(s_{j+1})\sin\phi_{j+1} + f(s_{j-1})\sin\phi_{j-1})] \\
                                  &-J_x(s_{j+1} + s_{j-1})
\end{align*}

\subsubsection{N = 4}

Using the approximations we had above

\begin{align*}
    \frac{d\delta s_j}{dt} &= \begin{cases}
        &J_y(\pm1)[\mp1 \pm 1] - J_z(\mp\delta\phi_j)[\mp\delta\phi_{j+1} \pm \delta\phi_{j-1}], j \text{ odd} \\
        &J_y(\pm\delta\phi_j)[\mp\delta\phi_{j+1} \pm \delta\phi_{j-1}] - J_z(\pm1)[\mp1 \pm 1], j \text{ even} \\
    \end{cases} \\
    &= \delta\phi_j\begin{cases}
        &J_z[\mp\delta\phi_{j+1} \pm \delta\phi_{j-1}], j \text{ odd} \\
        &J_y[\mp\delta\phi_{j+1} \pm \delta\phi_{j-1}], j \text{ even} \\
    \end{cases}
\end{align*}

\begin{align*}
    \frac{d\delta\phi_j}{dt} = -J_x(s_{j+1} + s_{j-1}) \mp \begin{cases}
        &J_z(\mp\phi_{j+1} \pm\phi_{j-1}), j \text{ odd} \\
        &J_y(\pm \phi_{j+1} \pm\phi_{j-1}), j \text{ even}
    \end{cases}
\end{align*}

\section{Numerics for Lyop Calc}
\section*{Comparison of Pushback Implementations in Benettin's Method}

The goal of the \texttt{push\_back} function in Benettin's algorithm is to restore the distance between two nearby trajectories, $\vec{S}_A$ (reference) and $\vec{S}_B$ (perturbed), to a fixed small value $\varepsilon$, without altering the direction of separation.

\subsection*{My Original Implementation}

In the original Julia implementation, the perturbation is applied \emph{locally} and independently to each spin:
\[
\vec{S}_{i}^{B} \gets \vec{S}_{i}^{A} + \varepsilon \cdot \frac{\vec{S}_{i}^{B} - \vec{S}_{i}^{A}}{||\vec{S}_{i}^{B} - \vec{S}_{i}^{A}||}
\]
for all $i = 1, \dots, L$.

That is, each spin is pushed away from its counterpart independently, ensuring that the local separation at each site is $\varepsilon$. However, this results in a total perturbation vector (in $\mathbb{R}^{3L}$) with norm approximately:
\[
||\vec{S}_B - \vec{S}_A|| \approx \varepsilon \cdot \sqrt{L}
\]

I tried adjusting the pushback to be

\[
\vec{S}_{i}^{B} \gets 1/\sqrt{L} (\vec{S}_{i}^{A} + \varepsilon \cdot \frac{\vec{S}_{i}^{B} - \vec{S}_{i}^{A}}{||\vec{S}_{i}^{B} - \vec{S}_{i}^{A}||})
\]

such that the total perturbation vector has norm

\[
||\vec{S}_B - \vec{S}_A|| \approx \varepsilon
\]

but that still didn't fix things.

\subsection*{Professor Pixely's Implementation (Benettin-consistent)}

In contrast, the correct implementation applies the perturbation to the entire spin chain \emph{as a single vector} in $\mathbb{R}^{3L}$:
\[
\vec{S}_B \gets \vec{S}_A + \varepsilon \cdot \frac{\vec{S}_B - \vec{S}_A}{||\vec{S}_B - \vec{S}_A||}
\]
This ensures that the total perturbation vector has norm exactly $\varepsilon$.

\textbf{Note: } I spoke with chatGPT and it also said that the correct implementation insured that the direction of perturbation \emph{is preserved} across steps. This consistency is essential for correctly capturing the dominant Lyapunov exponent. But I am not sure what is meant by that.
\[
\lambda = \lim_{n \to \infty} \frac{1}{n \tau} \sum_{i=1}^n \log \left( \frac{||\vec{S}_B^{(i)} - \vec{S}_A^{(i)}||}{\varepsilon} \right)
\]

\subsection*{Summary of Differences}

\begin{center}
\begin{tabular}{|l|c|c|}
\hline
\textbf{Feature} & \textbf{Original (local)} & \textbf{Professor's (global)} \\
\hline
Normalization per spin & Yes & No \\
Preserves global direction & No & Yes \\
Total perturbation norm & $\varepsilon \cdot \sqrt{L}$ & $\varepsilon$ \\
Valid for Benettin method & Not quite & Yes \\
\hline
\end{tabular}
\end{center}

\section{Harmonic Oscillators to Model Spin}

Consider a series RLC circuit driven by an AC voltage source:
\[
V_{\text{drive}}(t) = V_0 \cos(\omega_d t)
\]
The circuit contains:
\begin{itemize}
  \item Inductor with inductance $L$
  \item Resistor with resistance $R$
  \item Capacitor with capacitance $C$
\end{itemize}

\subsection{Equation of Motion}
\[
V_L + V_R + V_C = V_{\text{drive}}(t)
\]
\[
\boxed{L \ddot{Q} + R \dot{Q} + \frac{1}{C} Q = V_0 \cos(\omega_d t)}
\]

\subsection{Solution Structure}

The general solution is:
\[
Q(t) = Q_{\text{hom}}(t) + Q_{\text{part}}(t)
\]

The homogeneous solution describes transient behavior and decays over time due to resistance with time scale.

\[\tau = -\frac{2L}{R}\].

As in 

\[Q_{\text{hom}}(t) \sim e^{-\frac{Rt}{2L}}\]

I seek a steady state solution, so I feel like we should try something like:
\[
Q_{\text{part}}(t) = A \cos(\omega_d t) + B \sin(\omega_d t)
\]

\subsection{Determaning Particular Solution Ansatz}
Compute the first and second derivatives:
\[
\dot{Q}_{\text{part}}(t) = -A \omega_d \sin(\omega_d t) + B \omega_d \cos(\omega_d t)
\]
\[
\ddot{Q}_{\text{part}}(t) = -A \omega_d^2 \cos(\omega_d t) - B \omega_d^2 \sin(\omega_d t)
\]

Substitute into the differential equation:
\begin{align*}
&L \ddot{Q}_{\text{part}} + R \dot{Q}_{\text{part}} + \frac{1}{C} Q_{\text{part}} = V_0 \cos(\omega_d t) \\
&= L(-A \omega_d^2 \cos(\omega_d t) - B \omega_d^2 \sin(\omega_d t)) \\
&\quad + R(-A \omega_d \sin(\omega_d t) + B \omega_d \cos(\omega_d t)) \\
&\quad + \frac{1}{C}(A \cos(\omega_d t) + B \sin(\omega_d t))
\end{align*}

Group terms:
\[
\text{Coefficient of } \cos(\omega_d t):\quad -L A \omega_d^2 + R B \omega_d + \frac{A}{C}
\]
\[
\text{Coefficient of } \sin(\omega_d t):\quad -L B \omega_d^2 - R A \omega_d + \frac{B}{C}
\]

Set the equation equal to the driving term \( V_0 \cos(\omega_d t) \), and match coefficients:
\[
\begin{cases}
- L A \omega_d^2 + R B \omega_d + \frac{A}{C} = V_0 \\
- L B \omega_d^2 - R A \omega_d + \frac{B}{C} = 0
\end{cases}
\]

\subsection*{Solve the System}

We now solve this linear system for \( A \) and \( B \).

Define:
\[
X := \left( \frac{1}{C} - L \omega_d^2 \right), \quad Y := R \omega_d
\]

Then the system becomes:
\[
\begin{cases}
A X + B Y = V_0 \\
B X - A Y = 0
\end{cases}
\]

Solve the second equation for \( B \):
\[
B X = A Y \Rightarrow B = \frac{A Y}{X}
\]

Substitute into the first equation:
\[
A X + \left( \frac{A Y}{X} \right) Y = V_0
\Rightarrow A \left( X + \frac{Y^2}{X} \right) = V_0
\Rightarrow A = \frac{V_0 X}{X^2 + Y^2}
\]

Then:
\[
B = \frac{A Y}{X} = \frac{V_0 Y}{X^2 + Y^2}
\]

\subsection*{Final Particular Solution}

Thus, the particular solution is:
\[
Q_{\text{part}}(t) = \frac{V_0 X}{X^2 + Y^2} \cos(\omega_d t) + \frac{V_0 Y}{X^2 + Y^2} \sin(\omega_d t)
\]
\[
\text{where } X = \left( \frac{1}{C} - L \omega_d^2 \right), \quad Y = R \omega_d
\]

This can also be written in amplitude-phase form:
\[
Q_{\text{part}}(t) = Q_p \cos(\omega_d t - \delta)
\]

with:
\[
Q_p = \frac{V_0}{\sqrt{X^2 + Y^2}} = \frac{V_0}{\sqrt{\left( \frac{1}{C} - L \omega_d^2 \right)^2 + (R \omega_d)^2}}
\]
\[
\tan \delta = \frac{Y}{X} = \frac{R \omega_d}{\frac{1}{C} - L \omega_d^2}
\]

\subsection*{Voltage Across the Capacitor}

The voltage across the capacitor is:
\[
V_C(t) = \frac{Q_{\text{part}}(t)}{C} = \frac{Q_p}{C} \cos(\omega_d t - \delta)
\]

Final expression:
\[
\boxed{
V_C(t) = \frac{V_0}{\sqrt{(1 - L C \omega_d^2)^2 + (R C \omega_d)^2}} \cos(\omega_d t - \delta)
}
\quad \text{with} \quad
\tan \delta = \frac{R \omega_d}{\frac{1}{C} - L \omega_d^2}
\]

\end{document}